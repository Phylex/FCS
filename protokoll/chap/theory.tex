\section{FCS Basic Idea}
As previously stated, FCS observes the brightness of the radiation emitted from the probe
through the use of highly sensitive photo diodes. To distinguish the emitted radiation from the radiation used to illuminate the probe, the
frequencies of both light sources need to be different. This is accomplished by using specialized fluorophores specifically designed for the purpose
of FCS.
The underlying Idea of the FCS is to measure the fluctuations of the emitted radiation intensity rather that measuring the intensity itself. Through
autocorrelation of the signal, the time-scales of the different fluctuations can be measured along with the concentration of the emitting molecules in
the solution.
As the fluctuations of the intensity in comparison to the overall intensity reduce as the number of illuminated particles increases, it is very
important to keep the overall amount of illuminated particles below a certain number. As the number of illuminated molecules is described by the
poisson equation the fluctuations are proportional to it's uncertainty, which is:
\begin{equation}
	\delta F \propto \frac{\sqrt{\langle(\delta N)^2\rangle}}{\langle N\rangle}
	\label{eq:poisson_uncertainty}
\end{equation}
The fluctuations are defined as:
\begin{equation}
	\delta F(t) = F(t) - \langle F (t) \rangle = F(t) - \frac{1}{T}\int_0^TF(t)\text{d}t
	\label{eq:fluctuations}
\end{equation}

When using FCS, the amount of molecules that are illuminated are controlled by the size of the confocal volume. Only inside the confocal volume is the intensity of the illuminating
laser high enough to excite the fluorophores. As such the fluctuations of the intensity $\delta F$ can also be written as
\begin{equation}
	\delta F(t) = \kappa \int_V I_{ex}(\vec{r})S(\vec{r})\cdot \delta(\sigma q C(\vec{r},t))\text{d}V
	\label{eq:specail_fluctuations}
\end{equation}
where $\kappa$ is the 
